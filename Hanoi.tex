\chapter{Hanois tårne}
\section{Indledning}
Dette er en gammel kendt opgave formuleret af den franske matematiker NN.
Der er tale om et slags puslespil hvor man skal flytte nogle skiver mellem
3 pinde. Spillet findes også som legetøj. Spillet består af n skiver af
forskellig diameter. Disse er anbragt på en pind med største skive nederst
og derefter mindre og mindre skiver opad. På spillepladen findes yderligere
to pinde. Opgaven går herefter ud på at flytte alle skiverne fra en pind
til en anden stadig med den største skive nederst under den forudsætning,
at der flyttes en skive ad gangen og at der  i processen aldrig må placeres
en større skive ovenpå en mindre.

\section{Algoritme}
Vi ønsker nu at finde en generel måde at gøre dette på samt beregne minimum
antal flytninger der er nødvendige for at fuldføre opgaven.

Skiverne nummeres således at den mindste som ligger øverst får nummer \(1
\), den næstøverste nr. \(2\) og så videre op til nummer n, som er antallet
af skiver og nummeret på den nederste skive.
Den pind hvorpå skiverne ligger fra starten kalder vi \(A\) og de øvrige to
pinde kaldes henholdsvis \(B\) og \(C\). Antallet af nødvendige flytninger
kalder vi F(n).For at forstå processen prøver vi nu tilfældene \(n=1,2,3,4\).

\noindent \textbf{Tilfælde 1,\(n=1\):} Skive \(1\) fra pind \(A\) til pind \(B\). Hermed er
tårnet bestående af \(1\) skive flyttet og opgaven er fuldført. Så \(F(1)=1
\).

\noindent \textbf{Tilfælde 2,\(n=2\):} Skive \(1\) fra pind \(A\) til pind \(B\). Skive \(2\)
fra pind \(A\) til pind \(C\). Skive \(1\) fra pind \(B\) til pind \(C
\).Hermed er tårnet bestående af \(2\) skiver flyttet og opgaven er
fuldført. Så \(F(2)=3\).

\noindent \textbf{Tilfælde 3,\(n=3\):} Skive \(1\) fra pind \(A\) til pind \(B\). Skive \(2\)
fra pind \(A\) til pind \(C\). Skive \(1\) fra pind \(B\) til pind \(C
\).Skive \(3\) fra pind \(A\) til pind \(B\). Skive \(1\) fra pind \(C\)
til pind \(A\). Skive \(2\) fra pind \(C\) til pind \(B\).Skive \(1\) fra
pind \(A\) til pind \(B\).Hermed er tårnet bestående af \(3\) skiver
flyttet og opgaven er fuldført. Så \(F(3)=7\).

Bemærk i tilfælde \(3\), at man først flytter tårnet bestående af \(3-1=2\)
skiver til en ny pind \(C\). Dette tager ligeså mange flytninger som
tilfældet \(n=2\). Herefter flyttes sidste skive til pind \(B\). Til sidst
flyttes så tårnet med \(3-1=2\) skiver igen fra pind \(C\) til pind \(B\)
og opgaven er færdig.

Vi kan altså stykke processen op i 3 delaktioner:
\begin{itemize}
\item Flyt tårnet med \(n-1\) skiver til en ny pind. Dette tager \(F(n-1)\)
flytninger.
\item Flyt sidste pind - skive nr \(n\) - til en ny pind. Dette tager 1
flytning.
\item Flyt tårnet med  \(n-1\) skiver til pinden med skive nr \(n\).Dette
tager \(F(n-1)\) flytninger.
\end{itemize}

Hvis det tager \(F(n-1)\) flytninger for et tårn med \((n-1)\) skiver tager
det derfor \(2F((n-1)+1\) flytninger for et tårn med n skiver.  For at få
antallet af flytninger for et tårn med 1 skive ekstra skal man altså gange
antallet af flytninger for det forrige tårn med 2 og lægge en til, da vi jo
flytter tårnet med en skive mindre \(2\) gange og flytter den nederste
skive en gang. Vi kan nu stille en tabel op over antallet af flytninger for
de første værdier af \(n\).
\begin{align*}
&n \qquad &Flytninger&\ \\
&1  &1&=1\\
&2  &2 \times 1+1&=3\\
&3 &4 \times 1 + 2\times 1 + 1&=7\\
&4 &8 \times 1 + 4 \times 1+ 2 \times 1 +1&=15\\
&5 &16 \times 1 +8 \times 1 + 4\times 1 + 2 \times 1 +1&=31\\
\end{align*}
Det ses altså, at det totale antal flytninger er:
\begin{equation}
F(n) = 1+2+4+8+\ldots + 2^{(n-1)} = \sum_{k=1}^{n} 2^{(n-1)} = \sum_{k=0}^
{n-1}2^n
\end{equation}
Dette er en geometrisk række og der eksisterer en simpel løsning til denne
sum:
Vis sætter:
\begin{alignat*}{2}
&&S&=1+2+4+8+\ldots+2^{(n-1)}\\\\
&&2S&=2+4+8+16+\ldots+2\times 2^{(n-1)}\\
&&&=2+4+8+16+\ldots+2^n
\end{alignat*}
Den nederste ligning fremkommer ved af gange den øverste med 2. \(S\) er
den sum vi ønsker at finde. Trækker vi nu den øverste ligning fra den
nederste fås:
\[2S-S=(-1)+(2-2)+(4-4)+(8-8)\ldots+(2^{(n-1)}-2^{(n-1)})+2^n\]
Da alle led undtagen første og sidste går ud får vi:
\[F(n)=S=2^n-1\]
Dette er den søgte sum. For eksempel giver \(10\) skiver
\(2^{10-1}=1024-1=1023\) flytninger.

